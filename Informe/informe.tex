\documentclass{article}
\usepackage[utf8]{inputenc} %codificacion de caracteres que permite tildes
\usepackage{amsfonts}
\usepackage{natbib}
\usepackage{amsmath}
\usepackage{amssymb}
\usepackage{amsthm}
\usepackage{mathrsfs} % Cursive font
\usepackage{fancyhdr}
\usepackage{framed}
\usepackage{xspace}
\usepackage{dirtree}
\usepackage{multicol}
\usepackage{z-eves}

\usepackage[T1]{fontenc} % fastest font


\usepackage{float}
\usepackage{graphicx}
\usepackage{subcaption}

\newcommand{\desig}[2]{\item #1 $\approx #2$}
\newenvironment{designations}
  {\begin{leftbar}
    \begin{list}{}{\setlength{\labelsep}{0cm}
                   \setlength{\labelwidth}{0cm}
                   \setlength{\listparindent}{0cm}
                   \setlength{\rightmargin}{\leftmargin}}}
  {\end{list}\end{leftbar}}

\newcommand{\setlog}{$\{log\}$\xspace}


\usepackage{mathtools}
\usepackage{xparse} \DeclarePairedDelimiterX{\Iintv}[1]{\llbracket}{\rrbracket}{\iintvargs{#1}}
\NewDocumentCommand{\iintvargs}{>{\SplitArgument{1}{,}}m}
{\iintvargsaux#1}
\NewDocumentCommand{\iintvargsaux}{mm} {#1\mkern1.5mu,\mkern1.5mu#2}

\makeatletter
\newcommand*{\currentname}{\@currentlabelname}
\makeatother



\addtolength{\textwidth}{0.2cm}
\setlength{\parskip}{8pt}
\setlength{\parindent}{0.5cm}
\linespread{1.5}

\pagestyle{fancy}
\fancyhf{}
\rhead{Cipullo}
\lhead{TP Verificación de Software}
\rfoot{\vspace{1cm} \thepage}

\renewcommand*\contentsname{\LARGE Índice}

\setlength{\skip\footins}{0.5cm}


\begin{document}

\begin{titlepage}
    \hspace{-3.2cm}\includegraphics[scale= 0.48]{header.png}
    \begin{center}
        \vfill
            \noindent\textbf{\Huge Ingeniería de Software}\par
            \vspace{.5cm}
            \noindent\textbf{\Huge Trabajo Práctico Verificación de Software}\par
            \vspace{.5cm}
        \vfill
        % \noindent \textbf{\huge Alumnas:}\par
        \vspace{.5cm}
        \noindent \textbf{\Large Cipullo, Inés}\par
 
        \vfill
        % \large Universidad Nacional de Rosario \par
        \noindent\large 2025
    \end{center}
\end{titlepage}
\ 

\section{Requerimientos}

Se describen los requerimientos de un planificador de procesos a corto plazo, encargado de planificar los procesos que están listos para ejecución, también llamado \textit{dispatcher}. Este planificador implementará el algoritmo de planificaión ``Ronda'' (\textit{Round Robin} en inglés), y este sistema tendrá un único procesador.

Un proceso puede estar en alguno de los siguientes estados: \textit{nuevo}, \textit{listo}, \textit{ejecutando}, \textit{bloqueado}, \textit{terminado}. El dispatcher decide entre los procesos que están listos para ejecutarse y determina a cuál de ellos \textit{activar}, y detiene a aquellos que \textit{exceden su tiempo} de procesador, es decir, se encarga de las transiciones entre los estados \textbf{listo} y \textbf{ejecutando}. 

Siguiendo Round Robin, los procesos listos se almacenan en forma de cola, cada proceso listo se ejecuta por un sólo \textit{quantum} y si un proceso no ha terminado de ejecutarse al final de su período, será interrumpido y puesto al final de la cola de procesos listos.
Se deben tener en cuenta, también, aquellas transiciones que involucran otros estados de procesos pero inciden sobre alguno de los dos estados que se controlan desde el dispatcher. Los procesos que sean agregados a la cola de listos por estas otras transiciones, se ubicarán al final de la misma.


\section{Especificación}
Para empezar, se dan las siguientes designaciones.

\begin{designations}
\desig{$p$ es un proceso}{p \in PROCESS}
\desig{$t$ es un contador de ticks del sistema}{t \in TICK}
\desig{proceso nulo}{nullp}
\desig{cantidad de tiempo durante el cual un proceso tiene permiso para ejecutarse en el procesador antes de ser interrumpido}{quantum}
\desig{procesos en estado listo (para ser ejecutados)}{procQueue}
\desig{proceso en ejecución}{current}
\desig{contador de ticks restantes de ejecución que le corresponden a $current$}{remTicks}
\end{designations}

Luego, se introducen los tipos que se utilizan en la especificaci\'on. 

\begin{zed}
    [PROCESS]
\also
    TICK == \nat
\end{zed}

Adem\'as, se presentan las siguientes definiciones axiom\'aticas, donde se define la existencia del proceso $nullp$, que representa el proceso nulo, y la constante $quantum$, que equivale a 5 ticks.

\begin{axdef}
    nullp : PROCESS
\end{axdef}

\begin{axdef}
    quantum : TICK
\where
    quantum = 5
\end{axdef}

Se define entonces el espacio de estados del planificador y su estado inicial.

\begin{schema}{Dispatcher}
    procQueue : \nat \pinj PROCESS \\
    current : PROCESS \\
    remTicks : TICK
\end{schema}

\begin{schema}{InitDispatcher}
    Dispatcher
\where
    procQueue = \emptyset \\
    current = nullp \\
    remTicks = 0
\end{schema}

Como un proceso no puede estar \textit{en ejecución} y \textit{listo} al mismo tiempo, se plantea el siguiente invarinate de estado:

\begin{schema}{InvDispatcher}
    Dispatcher
\where
    current \notin \ran procQueue \\
    nullp \notin \ran procQueue
\end{schema}

Procedemos con la especificación de las operaciones requeridas. Estas son:
\begin{itemize}
    \item $NewProcess$: para pasar un proceso de estado \textit{nuevo} (o \textit{bloqueado}) a \textit{listo}.
    \item $Dispatch$: modela el funcionamiento del planificador, se encarga de las transiciones entre los estados \textit{listo} y \textit{en ejecución}.
    \item $TerminateProcess$: para pasar un proceso de estado \textit{en ejecución} a \textit{terminado}.
\end{itemize}

\begin{schema}{NewProcessOk}
    \Delta Dispatcher \\
    p? : PROCESS
\where
    procQueue \neq \emptyset \\
    p? \notin \ran procQueue \\
    p? \neq current \\
    p? \neq nullp \\
    procQueue' = procQueue \cup \{max~(\dom procQueue) + 1 \mapsto p?\} \\
    current' = current \\
    remTicks' = remTicks
\end{schema}

\begin{schema}{newProcessOkEmptyQueue}
    \Delta Dispatcher \\
    p? : PROCESS
\where
    procQueue = \emptyset \\
    p? \notin \ran procQueue \\
    p? \neq current \\
    p? \neq nullp \\
    procQueue' = \{1 \mapsto p?\} \\
    current' = current \\
    remTicks' = remTicks
\end{schema}

\begin{schema}{NewProcessError}
    \Xi Dispatcher \\
    p? : PROCESS
\where
    p? \in \ran procQueue \lor p? = current \lor p? = nullp
\end{schema}

\begin{zed}
    NewProcess == NewProcessOk \lor newProcessOkEmptyQueue \lor NewProcessError
\end{zed}

\begin{schema}{Tick}
    \Delta Dispatcher
\where
    remTicks > 0 \\
    current \neq nullp \\
    procQueue' = procQueue \\
    current' = current \\
    remTicks' = remTicks - 1    
\end{schema}

\begin{schema}{Timeout}
    \Delta Dispatcher
\where
    procQueue \neq \emptyset \\
    remTicks = 0 \\
    current \neq nullp \\
    procQueue' = procQueue \cup \{max~(\dom procQueue) + 1 \mapsto current\} \\
    current' = nullp \\
    remTicks' = remTicks 
\end{schema}

\begin{schema}{TimeoutEmptyQueue}
    \Delta Dispatcher
\where
    procQueue = \emptyset \\
    remTicks = 0 \\
    current \neq nullp \\
    procQueue' = \{1 \mapsto current\} \\
    current' = nullp \\
    remTicks' = remTicks 
\end{schema}

\begin{schema}{DispatchProcess}
    \Delta Dispatcher
\where
    current = nullp \\
    procQueue \neq \emptyset \\
    procQueue' = procQueue \setminus \{min~(\dom procQueue) \mapsto procQueue(min~(\dom procQueue))\} \\
    current' = procQueue(min~(\dom procQueue)) \\
    remTicks' = remTicks 
\end{schema}

\begin{schema}{Idle}
    \Xi Dispatcher
\where
    current = nullp \\
    procQueue = \emptyset
\end{schema}

\begin{zed}
    Dispatch == Tick \lor Timeout \lor TimeoutEmptyQueue \lor DispatchProcess \lor Idle
\end{zed}

\begin{schema}{TerminateProcessOk}
    \Delta Dispatcher \\
    p? : PROCESS
\where
    current = p? \\
    procQueue' = procQueue \\
    current' = nullp \\
    remTicks' = remTicks
\end{schema}

\begin{schema}{TerminateProcessError}
    \Xi Dispatcher \\
    p? : PROCESS
\where
    current \neq p?
\end{schema}

\begin{zed}
    TerminateProcess == TerminateProcessOk \lor TerminateProcessError
\end{zed}


\section{Simulaciones en \textit{\{log\}}}

Se traduce la especificación a \textit{\{log\}} (en \textbf{planificador.slog}), la cual pasa el \textbf{type\_check}. A continuación, se presentan dos simulaciones ejecutadas sobre \textit{\{log\}} y la primera solución de cada una.


\subsection*{1ra simulación}

\begin{verbatim}
    dec([PQ0, PQ1, PQ2, PQ3, PQ4, PQ5, PQ6, PQ7, PQ8, PQ9], t_proc_queue) &
    dec([C0, C1, C2, C3, C4, C5, C6, C7, C8, C9], process)                &
    dec([RT0, RT1, RT2, RT3, RT4, RT5, RT6, RT7, RT8, RT9], t_tick)       &
    initDispatcher(PQ0, C0, RT0)                                          & 
    newProcess(PQ0, C0, RT0, p1, PQ1, C1, RT1)                            & 
    newProcess(PQ1, C1, RT1, p2, PQ2, C2, RT2)                            & 
    dispatch(PQ2, C2, RT2, PQ3, C3, RT3)                                  & 
    dispatch(PQ3, C3, RT3, PQ4, C4, RT4)                                  & 
    dispatch(PQ4, C4, RT4, PQ5, C5, RT5)                                  & 
    dispatch(PQ5, C5, RT5, PQ6, C6, RT6)                                  & 
    dispatch(PQ6, C6, RT6, PQ7, C7, RT7)                                  & 
    dispatch(PQ7, C7, RT7, PQ8, C8, RT8)                                  & 
    dispatch(PQ8, C8, RT8, PQ9, C9, RT9).
\end{verbatim}

La idea de esta simulación es encolar dos nuevos procesos, \textbf{p1} y \textbf{p2}, en la cola de procesos listos, luego iniciar la ejecución del primero hasta que termine el tiempo de ejecución asignado (1 \textbf{quantum}) y sea interrumpido por el dispatcher. El resultado es el esperado:

\begin{verbatim}
    PQ0 = {},  
    C0 = nullp,  
    RT0 = 0,  
    PQ1 = {[1,p1]},  
    C1 = nullp,  
    RT1 = 0,  
    PQ2 = {[1,p1],[2,p2]},  
    C2 = nullp,  
    RT2 = 0,  
    PQ3 = {[2,p2]/_N1},  
    C3 = p1,  
    RT3 = 5,  
    PQ4 = {[2,p2]/_N1},  
    C4 = p1,  
    RT4 = 4,  
    PQ5 = {[2,p2]/_N1},  
    C5 = p1,  
    RT5 = 3,  
    PQ6 = {[2,p2]/_N1},  
    C6 = p1,  
    RT6 = 2,  
    PQ7 = {[2,p2]/_N1},  
    C7 = p1,  
    RT7 = 1,  
    PQ8 = {[2,p2]/_N1},  
    C8 = p1,  
    RT8 = 0,  
    PQ9 = {[3,p1],[2,p2]/_N1},  
    C9 = nullp,  
    RT9 = 0
    Constraint: subset(_N1,{[1,p1],[2,p2]}), [1,p1]nin _N1, 
    set(_N1), dom(_N1,_N2), set(_N2), foreach(_X in _N2,2>=_X), 
    rel(_N1)
\end{verbatim}


\subsection*{2da simulación}

\begin{verbatim}
    initDispatcher(PQ0, C0, RT0)                  & 
    newProcess(PQ0, C0, RT0, p1, PQ1, C1, RT1)    & 
    newProcess(PQ1, C1, RT1, p2, PQ2, C2, RT2)    & 
    dispatch(PQ2, C2, RT2, PQ3, C3, RT3)          & 
    dispatch(PQ3, C3, RT3, PQ4, C4, RT4)          & 
    terminateProc(PQ4, C4, RT4, p1, PQ5, C5, RT5) & 
    dispatch(PQ5, C5, RT5, PQ6, C6, RT6). 
\end{verbatim}

La idea de esta simulación es encolar dos nuevos procesos, \textbf{p1} y \textbf{p2}, en la cola de procesos listos, luego iniciar la ejecución del primero, que ejecute por un tick y que termine su ejecución. Despues se inicia la ejecución del segundo. El resultado es el esperado:

\begin{verbatim}
    PQ0 = {},  
    C0 = nullp,  
    RT0 = 0,  
    PQ1 = {[1,p1]},  
    C1 = nullp,  
    RT1 = 0,  
    PQ2 = {[1,p1],[2,p2]},  
    C2 = nullp,  
    RT2 = 0,  
    PQ3 = {[2,p2]/_N1},  
    C3 = p1,  
    RT3 = 5,  
    PQ4 = {[2,p2]/_N1},  
    C4 = p1,  
    RT4 = 4,  
    PQ5 = {[2,p2]/_N1},  
    C5 = nullp,  
    RT5 = 0,  
    C6 = p2,  
    RT6 = 5
    Constraint: subset(_N1,{[1,p1],[2,p2]}), [1,p1]nin _N1, 
    set(_N1), dom(_N1,_N2), set(_N2), foreach(_X in _N2,2=<_X), 
    pfun(_N1), comppf({[2,2]},_N1,{}), subset(PQ6,{[2,p2]/_N1}), 
    subset(_N1,{[2,p2]/PQ6}), [2,p2]nin PQ6, rel(_N1), set(PQ6)
\end{verbatim}


\section{Generado de Condiciones de Verificación}

Se utiliza el VCG para generar condiciones de verificación sobre la especificación en \textit{\{log\}}, lo cual genera el archivo \textbf{planificador-vc.slog}. Luego el comando \verb|check_vcs_planificador| realiza todas las descargas de prueba automaticamente.

\subsection*{Sobre la interacción con el VCG}

En una primera iteración del comando \verb|check_vcs_planificador|, hubo algunas condiciones de verificación que no fueron descargadas exitosamente. Por lo que fue necesario agregar las siguientes hipótesis en las condición de verificación que lo requerían:

\begin{itemize}
    \item hipótesis \verb|invDispatcher(ProcQueue, Current)| en \verb|dispatch_pi_invIny|
    \item hipótesis \verb|invIny(ProcQueue)| en \verb|dispatch_pi_invDispatcher|
\end{itemize}

En ambos casos plantee la negación del enunciado y analicé el contraejemplos que presentaba para saber que hipótesis era necesario agregar. No fue necesario utilizar el comando \verb|findh|.

También fue necesario aumentar el timeout para que se logren descargar algunas pruebas, esto lo hice modificando el argumento de \verb|def_to|.

\section{Demostración en \textit{Z/EVES}}

Utilizando el asistente de pruebas \textit{Z/EVES}, vamos a demostrar que la operación \verb|TerminateProcess| preserva el invariante de estado \verb|InvDispatcher|.

\begin{theorem}{TerminateProcessInvDispatcher}
    InvDispatcher \land TerminateProcess \implies InvDispatcher'
\end{theorem}

\begin{zproof}[TerminateProcessInvDispatcher]
    invoke TerminateProcess;
    split TerminateProcessOk;
    simplify;
    cases;
    invoke TerminateProcessOk;
    invoke InvDispatcher;
    equality substitute procQueue';
    simplify;
    next;
    invoke TerminateProcessError;
    invoke InvDispatcher;
    invoke \Xi Dispatcher;
    rewrite;
    next;
\end{zproof}

\section{Casos de Prueba con \textsc{Fastest}}

Se generan casos de prubea utilizando la herramienta \textsc{Fastest} para la operación \verb|NewProcess|.

\subsection*{Comandos utilizados}

\begin{verbatim}
    loadspec ../fastest.tex
    selop NewProcess
    genalltt
    addtactic NewProcess_DNF_1 SP \notin p? \notin \ran procQueue
    addtactic NewProcess_DNF_3 SP \in p? \in \ran procQueue
    genalltt
    genalltca
\end{verbatim}

Con estos comandos se carga la especificación, se selecciona la operación para la cual generar los casos de prueba y se aplican las tacticas de testing.
En primer lugar, se aplica \textit{Disjunctive Normal Form (DNF)}, que lleva la operación a su forma normal disyuntiva, ya que es la táctica por defecto que aplica \textsc{Fastest} al ejecutar el comando \verb|genalltt|. 
Esto particiona la operación de acuerdo a las precondiciones de las suboperaciones que la definen, donde \textit{NewProcess\_DNF\_1} corresponde a \textit{newProcessOk}, \textit{NewProcess\_DNF\_2} corresponde a \textit{newProcessOkEmptyQueue} y las tres restantes corresponden a las tres alternativas de \textit{newProcessError}.
Luego, se aplican las tacticas de partición estándar (\textit{Standard Partition (SP)}) de $\not\in$ y $\in$, donde resulta útil. Se obtiene el siguiente árbol:

\begin{minipage}{5cm}
\dirtree{%
    .1 $NewProcess\_VIS$.
    .2 $NewProcess\_DNF\_1$.
    .3 $NewProcess\_SP\_1$.
    .3 $NewProcess\_SP\_2$.
    .2 $NewProcess\_DNF\_2$.
    .2 $NewProcess\_DNF\_3$.
    .3 $NewProcess\_SP\_3$.
    .3 $NewProcess\_SP\_4$.
    .2 $NewProcess\_DNF\_4$.
    .2 $NewProcess\_DNF\_5$.
}
\end{minipage}

Finalmente se generan los casos de prueba, obteniendo el siguiente árbol de clases de prueba:

\begin{minipage}{7cm}
\dirtree{%
    .1 $NewProcess\_VIS$.
    .2 $NewProcess\_DNF\_1$.
    .3 $NewProcess\_SP\_2$.
    .4 $NewProcess\_SP\_2\_TCASE$.
    .2 $NewProcess\_DNF\_2$.
    .3 $NewProcess\_DNF\_2\_TCASE$.
    .2 $NewProcess\_DNF\_3$.
    .3 $NewProcess\_SP\_3$.
    .4 $NewProcess\_SP\_3\_TCASE$.
    .3 $NewProcess\_SP\_4$.
    .4 $NewProcess\_SP\_4\_TCASE$.
    .2 $NewProcess\_DNF\_4$.
    .3 $NewProcess\_DNF\_4\_TCASE$.
    .2 $NewProcess\_DNF\_5$.
    .3 $NewProcess\_DNF\_5\_TCASE$.
}
\end{minipage}

Notamos que no se generó un caso de prueba para la hoja \verb|NewProcess_SP_1|. Esto es porque es una clase de prueba vacía, ya que contiene tanto $procQueue \neq \{\}$ y $procQueue = \{\}$ como precondiciones.

Por último, se muestran los esquemas de los casos de prueba abstractos generados por \textsc{Fastest}.


\begin{schema}{NewProcess\_ VIS}\\
procQueue : \nat \pinj PROCESS \\
current : PROCESS \\
remTicks : TICK \\
p? : PROCESS 
\where
( procQueue \neq \{ \} \\
p? \notin \ran procQueue \\
p? \neq current \\
p? \neq nullp ) \lor ( procQueue = \{ \} \\
p? \notin \ran procQueue \\
p? \neq current \\
p? \neq nullp ) \lor p? \in \ran procQueue \lor p? = current \lor p? = nullp
\end{schema}

\begin{multicols}{2}

\begin{schema}{NewProcess\_ DNF\_ 1}\\
NewProcess\_ VIS 
\where
procQueue \neq \{ \} \\
p? \notin \ran procQueue \\
p? \neq current \\
p? \neq nullp
\end{schema}

\begin{schema}{NewProcess\_ SP\_ 1}\\
NewProcess\_ DNF\_ 1 
\where
\ran procQueue = \{ \}
\end{schema}

\begin{schema}{NewProcess\_ SP\_ 2}\\
NewProcess\_ DNF\_ 1 
\where
\ran procQueue \neq \{ \}
\end{schema}

\begin{schema}{NewProcess\_ DNF\_ 2}\\
NewProcess\_ VIS 
\where
procQueue = \{ \} \\
p? \notin \ran procQueue \\
p? \neq current \\
p? \neq nullp
\end{schema}

\begin{schema}{NewProcess\_ DNF\_ 3}\\
NewProcess\_ VIS 
\where
p? \in \ran procQueue
\end{schema}

\begin{schema}{NewProcess\_ SP\_ 3}\\
NewProcess\_ DNF\_ 3 
\where
\ran procQueue = \{ p? \}
\end{schema}

\begin{schema}{NewProcess\_ SP\_ 4}\\
NewProcess\_ DNF\_ 3 
\where
\ran procQueue \neq \{ p? \} \\
p? \in \ran procQueue
\end{schema}

\begin{schema}{NewProcess\_ DNF\_ 4}\\
NewProcess\_ VIS 
\where
p? = current
\end{schema}

\begin{schema}{NewProcess\_ DNF\_ 5}\\
NewProcess\_ VIS 
\where
p? = nullp
\end{schema}

\begin{schema}{NewProcess\_ SP\_ 2\_ TCASE}\\
NewProcess\_ SP\_ 2
\end{schema}

\begin{schema}{NewProcess\_ DNF\_ 2\_ TCASE}\\
NewProcess\_ DNF\_ 2
\end{schema}

\begin{schema}{NewProcess\_ SP\_ 3\_ TCASE}\\
NewProcess\_ SP\_ 3
\end{schema}

\begin{schema}{NewProcess\_ SP\_ 4\_ TCASE}\\
NewProcess\_ SP\_ 4
\end{schema}

\begin{schema}{NewProcess\_ DNF\_ 4\_ TCASE}\\
NewProcess\_ DNF\_ 4
\end{schema}

\begin{schema}{NewProcess\_ DNF\_ 5\_ TCASE}\\
NewProcess\_ DNF\_ 5
\end{schema}

\end{multicols}

\end{document}

