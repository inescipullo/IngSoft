\documentclass{article}
\usepackage[utf8]{inputenc} %codificacion de caracteres que permite tildes
% \usepackage[spanish]{babel}

\usepackage{amsfonts}
\usepackage{natbib}
\usepackage{amsmath}
\usepackage{amssymb}
\usepackage{amsthm}
\usepackage{mathrsfs} % Cursive font
% \usepackage{ragged2e}
\usepackage{fancyhdr}
% \usepackage{nameref}
% \usepackage{wrapfig}
\usepackage{hyperref}
\usepackage{framed}
\usepackage{xspace}
\usepackage{dirtree}
\usepackage{multicol}
\usepackage{z-eves}


\usepackage{float}
\usepackage{graphicx}
\usepackage{subcaption}
% \graphicspath{ {./Resources/} }

% \usepackage[
% top    = 2cm,
% bottom = 1.5cm,
% left   = 1.5cm,
% right  = 1.5cm]
% {geometry}

\newcommand{\desig}[2]{\item #1 $\approx #2$}
\newenvironment{designations}
  {\begin{leftbar}
    \begin{list}{}{\setlength{\labelsep}{0cm}
                   \setlength{\labelwidth}{0cm}
                   \setlength{\listparindent}{0cm}
                   \setlength{\rightmargin}{\leftmargin}}}
  {\end{list}\end{leftbar}}

\newcommand{\setlog}{$\{log\}$\xspace}


\usepackage{mathtools}
\usepackage{xparse} \DeclarePairedDelimiterX{\Iintv}[1]{\llbracket}{\rrbracket}{\iintvargs{#1}}
\NewDocumentCommand{\iintvargs}{>{\SplitArgument{1}{,}}m}
{\iintvargsaux#1}
\NewDocumentCommand{\iintvargsaux}{mm} {#1\mkern1.5mu,\mkern1.5mu#2}

\makeatletter
\newcommand*{\currentname}{\@currentlabelname}
\makeatother



\addtolength{\textwidth}{0.2cm}
\setlength{\parskip}{8pt}
\setlength{\parindent}{0.5cm}
\linespread{1.5}

\pagestyle{fancy}
\fancyhf{}
\rhead{Cipullo}
\lhead{TP Verificación de Software}
\rfoot{\vspace{1cm} \thepage}

\renewcommand*\contentsname{\LARGE Índice}

\setlength{\skip\footins}{0.5cm}


\begin{document}

\begin{titlepage}
    \hspace{-2.5cm}\includegraphics[scale= 0.48]{header.png}
    \begin{center}
        \vfill
            \noindent\textbf{\Huge Ingeniería de Software}\par
            \vspace{.5cm}
            \noindent\textbf{\Huge Trabajo Práctico Verificación de Software}\par
            \vspace{.5cm}
        \vfill
        % \noindent \textbf{\huge Alumnas:}\par
        \vspace{.5cm}
        \noindent \textbf{\Large Cipullo, Inés}\par
 
        \vfill
        % \large Universidad Nacional de Rosario \par
        \noindent\large 2025
    \end{center}
\end{titlepage}
\ 

\section{Requerimientos}

Se describen los requerimientos de un planificador de procesos a corto plazo, encargado de planificar los procesos que están listos para ejecución, también llamado \textit{dispatcher}. Este planificador implementará el algoritmo de planificaión ``Ronda'' (\textit{Round Robin} en inglés), y este sistema tendrá un único procesador.

Un proceso puede estar en alguno de los siguientes estados: \textit{nuevo}, \textit{listo}, \textit{ejecutando}, \textit{bloqueado}, \textit{terminado}. El dispatcher decide entre los procesos que están listos para ejecutarse y determina a cuál de ellos \textit{activar}, y detiene a aquellos que \textit{exceden su tiempo} de procesador, es decir, se encarga de las transiciones entre los estados \textbf{listo} y \textbf{ejecutando}. 

Siguiendo Round Robin, los procesos listos se almacenan en forma de cola, cada proceso listo se ejecuta por un sólo \textit{quantum} y si un proceso no ha terminado de ejecutarse al final de su período, será interrumpido y puesto al final de la cola de procesos listos.
Se deben tener en cuenta, también, aquellas transiciones que involucran otros estados de procesos pero inciden sobre alguno de los dos estados que se controlan desde el dispatcher. Los procesos que sean agregados a la cola de listos por estas otras transiciones, se ubicarán al final de la misma.


\section{Especificación}
Para empezar, se dan las siguientes designaciones.

\begin{designations}
\desig{$p$ es un proceso}{p \in PROCESS}
\desig{$t$ es un contador de ticks del sistema}{t \in TICK}
\desig{proceso nulo}{nullp}
\desig{cantidad de tiempo durante el cual un proceso tiene permiso para ejecutarse en el procesador antes de ser interrumpido}{quantum}
\desig{cola de procesos listos para ser ejecutados}{procQueue}
\desig{proceso en ejecución}{current}
\desig{contador de ticks restantes de ejecución que le corresponden a $current$}{remTicks}
\end{designations}

Luego, se introducen los tipos que se utilizan en la especificaci\'on. 

\begin{zed}
    [PROCESS]
\also
    TICK ::= \nat
\end{zed}

Adem\'as, se presentan las siguientes definiciones axiom\'aticas, donde se define la existencia del proceso $nullp$, que representa el proceso nulo, y la constante $quantum$, que equivale a 5 ticks.

\begin{axdef}
    nullp : PROCESS
\end{axdef}

\begin{axdef}
    quantum : TICK
\where
    quantum = 5
\end{axdef}

Se define entonces el espacio de estados del planificador y su estado inicial.

\begin{schema}{Dispatcher}
    procQueue : \seq \langle PROCESS \rangle \\
    current : PROCESS \\
    remTicks : TICK
\end{schema}

\begin{schema}{InitDispatcher}
    Dispatcher
\where
    procQueue = \langle \rangle \\
    current = nullp \\
    remTicks = 0
\end{schema}

Como un proceso no puede estar \textit{en ejecución} y \textit{listo} al mismo tiempo, se plantea el invarinate de que $current$ no puede pertenecer a $procQueue$.

\begin{schema}{InvDispatcher}
    Dispatcher
\where
    procQueue \rres \{current\} = \emptyset
\end{schema}

Procedemos con la especificación de las operaciones requeridas. Estas son:
\begin{itemize}
    \item $NewProcess$: para pasar un proceso de estado \textit{nuevo} (o \textit{bloqueado}) a \textit{listo}.
    \item $Dispatch$: modela el funcionamiento del planificador, se encarga de las transiciones entre los estados \textit{listo} y \textit{en ejecución}.
    \item $TerminateProcess$: para pasar un proceso de estado \textit{en ejecución} a \textit{terminado}.
\end{itemize}

\begin{schema}{NewProcessOk}
    \delta Dispatcher \\
    p? : PROCESS
\where
    procQueue \rres \{p?\} = \emptyset \\
    current \neq p? \\
    procQueue' = procQueue \cat \langle p? \rangle \\
    current' = current \\
    remTicks' = remTicks
\end{schema}

\begin{schema}{NewProcessError}
    \Xi Dispatcher
    p? : PROCESS
\where
    procQueue \rres \{p?\} \neq \emptyset \lor current = p?
\end{schema}

\begin{zed}
    NewProcess == NewProcessOk \lor NewProcessError
\end{zed}

\begin{schema}{Tick}
    \Delta Dispatcher
\where
    remTicks > 0 \\
    current \neq nullp \\
    procQueue' = procQueue \\
    current' = current \\
    remTicks' = remTicks - 1    
\end{schema}

\begin{schema}{Timeout}
    \Delta Dispatcher
\where
    remTicks = 0 \\
    current \neq nullp \\
    procQueue' = procQueue \cat \langle current \rangle \\
    current' = nullp \\
    remTicks' = remTicks 
\end{schema}

\begin{schema}{DispatchProcess}
    \Delta Dispatcher
\where
    current = nullp \\
    procQueue \neq \langle \rangle \\
    procQueue' = procQueue \cat \langle current \rangle \\
    current' = nullp \\
    remTicks' = remTicks 
\end{schema}

\begin{schema}{Idle}
    \Xi Dispatcher
\where
    current = nullp \\
    procQueue = \langle \rangle
\end{schema}

\begin{zed}
    Dispatch == Tick \lor Timeout \lor DispatchProcess \lor Idle
\end{zed}

\begin{schema}{TerminateProcessOk}
    \Delta Dispatcher \\
    p? : PROCESS
\where
    current = p? \\
    procQueue' = procQueue \\
    current' = nullp \\
    remTicks' = remTicks
\end{schema}

\begin{schema}{TerminateProcessError}
    \Xi Dispatcher \\
    p? : PROCESS
\where
    current \neq p?
\end{schema}

\begin{zed}
    TerminateProcess == TerminateProcessOk \lor TerminateProcessError
\end{zed}


\end{document}